\chapter{Introduzione}
\lettrine[lines=2]{L}{'} avanzamento tecnologico ha trasformato radicalmente la società spingendola verso l'esplorazione di nuove frontiere nell'automazione e nell'interazione digitale. Un'enfasi particolare è stata posta sulla localizzazione di oggetti e persone in ambienti interni, facilitando lo sviluppo di una gamma estesa di servizi e applicazioni innovative. In questo contesto, la localizzazione indoor, definita come il processo di identificazione della posizione di dispositivi o individui in spazi chiusi, emerge come un campo di ricerca e sviluppo particolarmente vivace. 

Originariamente focalizzata su settori quali l'industria e la robotica, l'importanza della localizzazione indoor ha guadagnato nuove prospettive con l'espansione dei dispositivi mobili dotati di tecnologie di comunicazione senza fili. Questo cambiamento ha associato la localizzazione e il tracciamento degli utenti alla posizione dei loro dispositivi, aprendo nuove possibilità nell'Internet delle Cose (IoT), nelle città intelligenti, negli edifici connessi e oltre.

I sistemi di posizionamento globale (GPS), nonostante abbiano rivoluzionato la localizzazione all'aperto, trovano limitazioni in ambienti chiusi a causa di segnali deboli e imprecisioni\cite{bisio_pervasive_2011}. La localizzazione indoor non rappresenta solamente un vantaggio tecnologico, ma soddisfa una necessità critica in vari ambiti, tra cui il settore manifatturiero e quello sanitario, dove può incrementare notevolmente l'efficienza e la sicurezza tracciando attrezzature e personale. È inoltre fondamentale per la sicurezza pubblica, facilitando l'evacuazione sicura degli edifici in caso di emergenza\cite{bisio_wifi_barometer_2018}. Nonostante ciò, rimangono sfide significative relative alla precisione, copertura e gestione energetica. Tecnologie come Bluetooth, Zigbee, Wi-Fi e Ultra Wide Band (UWB) costituiscono le fondamenta delle soluzioni attuali, inserendosi nell'ampio ecosistema dell'IoT.

\noindent Questa ricerca esamina come l'evoluzione dei dispositivi mobili e dell'IoT si intrecci con le tecnologie di localizzazione indoor, evidenziando come \hl{(in che modo?)}.la convergenza tra dispositivi multifunzionali e avanzate tecnologie di comunicazione stia modellando il futuro della localizzazione in spazi chiusi. Attraverso un'analisi dettagliata delle capacità, dei limiti e delle potenzialità di diverse tecnologie, il lavoro mira a tracciare strategie per affrontare le sfide esistenti, proponendo soluzioni per realizzare sistemi di localizzazione indoor più precisi, affidabili e diffusi. Un'attenzione speciale è rivolta all'applicazione di queste tecnologie per la localizzazione di muletti in contesti industriali, con l'obiettivo di condurre test empirici per valutare l'efficacia delle soluzioni proposte.

\noindent In particolare, la ricerca si concentra sull'importanza di localizzare con precisione veicoli industriali, come i muletti, per ottimizzare la logistica, incrementare la sicurezza e migliorare l'efficienza operativa. Si esplora come diverse tecnologie di localizzazione possano essere applicate e integrate per superare questa sfida, analizzando i risultati dei test empirici effettuati direttamente su un muletto per offrire una comprensione approfondita delle prestazioni delle tecnologie in condizioni reali. Inoltre, si discute dell'importanza della collaborazione tra tecnologie a corto e medio/lungo raggio per realizzare una localizzazione continua e ubiqua nei futuri network e servizi IoT, sia in ambienti statici che in movimento. 

A partire da tali premesse, il lavoro di tesi si propone di offrire una panoramica completa che non solo metta in luce i progressi raggiunti ma che identifichi anche le aree che necessitano di ulteriore ricerca e sviluppo. Con un focus specifico sulla localizzazione di veicoli industriali, il lavoro aspira a guidare l'innovazione verso soluzioni di localizzazione indoor più integrate ed efficienti, in grado di soddisfare le esigenze di un mondo sempre più interconnesso e automatizzato.